\cleardoublepage

\section{背景知识与技术}

在探讨服务监管语言的设计之前,有必要了解一些关键的背景知识和技术,这些将为后续的研究提供理论和实践基础。

\subsection{服务监管系统架构}

\begin{figure}[ht]
    \centering
    \includegraphics[width=.4\linewidth]{logo/zju}
    \caption{\label{fig:total-struct}服务监管系统架构}
\end{figure}

服务监管语言是构建高效服务监管系统的核心要素,它在整个监管框架中扮演着至关重要的角色。\autoref{fig:total-struct}描述了服务监管系统的工作流程和架构概览,具体包括:

\begin{itemize}
    \item \textbf{数据集的整合与预处理:} 系统从多模态的规则数据集着手,这些数据集可能包含文本、语音、图像等多种格式的规则信息。
    \item \textbf{规则数据库的初步构建:} 通过应用服务监管语言,原始的多模态规则数据被转换成一种半结构化的格式,进而形成规则数据库。这一步骤是将非结构化数据转换为机器可读格式的关键环节。
    \item \textbf{语法规范与处理:} 遵守既定的语法规范,系统对半结构化的规则数据进行进一步的精细化处理,提炼出规则的深层次结构,并构建成语法树。语法树的构建为后续的语义分析和规则执行提供了清晰的逻辑框架。
    \item \textbf{语义建模与检测:} 在语法树的基础上,系统进行语义建模,将规则中的自然语言转化为机器可理解的语义表达。同时,系统执行语义检测,确保规则的一致性和无歧义性,避免潜在的逻辑矛盾。
    \item \textbf{大模型服务监管的实现:} 最后,经过上述步骤处理的规则被应用于大模型驱动的服务监管工作中。大模型凭借其强大的数据处理和智能识别能力,能够高效地执行监管任务,实现服务的自动化和智能化监管。
\end{itemize}

整个服务监管系统的架构设计旨在实现规则的高效管理、精准执行和智能监管。服务监管语言的设计和应用,确保了规则从原始数据到最终执行的每一步都准确无误,为服务监管的自动化和智能化提供了坚实的基础。通过这种系统化的方法,我们能够应对日益增长的监管需求,提高监管的效率和效果,确保服务的合规性和安全性。

\subsection{Antlr语法分析器生成工具}

Antlr,全称为“ANother Tool for Language Recognition”,是一款功能强大的语法分析器生成工具,主要用于从语言规范中读取构造语法分析器或者转译器的代码。它有以下几个主要特点:

\begin{itemize}
    \item \textbf{易用性:} Antlr提供了一种简洁、直观的语法表示方式,使得用户能够快速地定义和理解语法规则。
    \item \textbf{强大的解析能力:} Antlr能够处理广泛的语法结构,包括LL(*)和LR(k)等。无论是处理简单的文本格式,还是复杂的编程语言,Antlr都能够应对自如。
    \item \textbf{丰富的目标语言支持:} Antlr支持生成多种编程语言的解析器,包括Java、C++、Python、JavaScript等,满足了用户的多样化需求。
    \item \textbf{强大的错误处理机制:} Antlr具有强大的错误处理和恢复机制,使得解析器能够在出错时提供有用的反馈,帮助用户识别和修复问题。
    \item \textbf{易于集成:} Antlr生成的解析器易于集成到用户的应用程序中,无论是作为一个独立的组件,还是作为一个库,都能够轻松实现。
\end{itemize}

Antlr是一款功能强大、易用的语法分析器生成工具,无论是用于编译原理的学习,还是用于实际的语言设计和实现,都是一个理想的选择。

\subsection{大语言模型}
大语言模型(Large Language Models,LLMs)在自然语言处理(NLP)领域中是一个重大的突破。它们代表了深度学习在模拟人类语言理解方面所取得的进展。以下是大语言模型的一些核心特性:

\begin{itemize}
    \item \textbf{规模与复杂性:} 大语言模型利用大量的参数(通常是数十亿到数千亿的数量级)来捕获和理解语言的复杂性。这种规模的模型能够学习到语言的深层结构和细微差别,从而在各种语言任务中展现出优异的性能。
    \item \textbf{预训练与微调:} 这些模型通常首先在大规模的数据集上进行预训练,学习语言的通用表示。预训练阶段使模型能够积累广泛的知识,之后在特定任务上的微调则进一步提升了模型的专业性,优化其性能。
    \item \textbf{上下文理解:} 大语言模型能够理解文本的上下文信息,这使得它们能够生成连贯且富有深度的文本内容。这种上下文感知能力是实现高级语言任务,如理解隐喻、同义等复杂结构的关键。
    \item \textbf{多任务学习:} 它们具有处理多种不同任务的能力,如文本分类、问答、文本生成等。这种多任务学习的能力使得大语言模型可以灵活地应用于广泛的应用场景,显示出了令人印象深刻的适应性和灵活性。
    \item \textbf{生成能力:} 大语言模型不仅能够理解语言,还能够生成语言。这使得它们在创意写作、内容生成和对话系统等领域具有广泛的应用潜力。
\end{itemize}

大语言模型在服务监管语言的设计和实施中发挥了关键作用。它们可以用于增强语言的自动解析能力,提升对语义的理解,并执行复杂的逻辑推理任务。此外,大模型还能够辅助监管者在制定和执行规则时更好地理解和应用语言的细微差别,从而提高服务监管的准确性和效率。

\subsection{随机布尔可满足性}
随机布尔可满足性(Stochastic Boolean Satisfiability, SSAT)是逻辑和计算机科学中的一个概念,它在处理不确定性和概率性问题时非常有用。SSAT是传统布尔可满足性问题(Boolean Satisfiability Problem, SAT)的扩展,其中引入了概率因素。以下是SSAT的一些核心概念和应用:

\begin{itemize}
    \item \textbf{概率模型:} 在SSAT中,每个布尔变量被赋予一个满足概率,即该变量为真的可能性。这种概率化表示使得SSAT能够模拟现实世界中的不确定性。
    \item \textbf{逻辑约束:} 与传统的SAT问题类似,SSAT中的公式由布尔逻辑运算符(如AND、OR和NOT)连接的变量组成。不同之处在于,SSAT考虑了每个变量的满足概率。
    \item \textbf{可满足性计算:} SSAT的目标是找到一组变量赋值,使得整个公式的满足概率最大化,或者计算在给定约束下公式满足的概率。
    \item \textbf{应用领域:} SSAT在多个领域都有应用,包括机器学习、人工智能、风险评估和决策支持系统。它特别适用于需要在不确定性下做出决策的场景。
    \item \textbf{算法和求解器:} 解决SSAT问题通常需要专门的算法和求解器。这些工具可以处理复杂的逻辑公式,并计算出满足特定概率阈值的解决方案。
    \item \textbf{与服务监管的联系:} 在服务监管语言的设计中,SSAT可以用来评估监管规则在不同情况下的满足概率,帮助监管者理解规则的潜在影响,并优化规则以提高监管效率。
\end{itemize}

SSAT作为一种强大的工具,为服务监管语言提供了一种评估和优化监管规则的新方法。通过考虑概率因素,监管者可以更全面地理解和应对监管环境中的不确定性。

\subsection{Z3求解器}
Z3是一个由Microsoft Research开发的高效SMT(Satisfiability Modulo Theories)求解器,它在形式化验证、程序验证和符号计算等多个领域有着广泛的应用。以下是Z3求解器的一些关键特性:

\begin{itemize}
    \item \textbf{多种逻辑支持:} Z3支持多种逻辑,包括但不限于二阶逻辑、量化逻辑和各种片段的一阶逻辑,使其能够解决广泛的数学和逻辑问题。
    \item \textbf{高性能:} Z3采用了许多先进的优化技术,如按需模型生成、高效的搜索策略和并行处理,以提高求解速度和效率。
    \item \textbf{可扩展性:} 设计上注重可扩展性,Z3能够处理大规模的问题实例,同时保持相对合理的性能。
    \item \textbf{集成与API:} 提供了丰富的API,支持多种编程语言,包括C++、Python、Scala等,方便开发者将其集成到不同的应用中。
    \item \textbf{应用场景:} Z3被用于软件和硬件验证、网络安全、优化问题、机器学习模型的验证以及形式化方法教育等领域。
    \item \textbf{定理证明:} 在定理证明中,Z3可以自动化地证明给定的公式或者查找使其为假的反例,大大简化了证明过程。
    \item \textbf{挑战与限制:} 尽管Z3非常强大,但它也有局限性,如处理某些特定类型问题时的性能瓶颈,以及对于非专业人员来说较高的使用门槛。
\end{itemize}

Z3求解器在服务监管语言的实现中扮演了重要角色,特别是在需要形式化验证和逻辑推理的场景中。通过使用Z3,开发者可以确保监管规则的逻辑正确性,并自动化地验证规则之间的一致性和完整性。

\subsection{Sentence Transformers}
Sentence Transformers是由德国柏林工业大学的团队开发的,它们基于预训练的语言模型,如BERT或RoBERTa,用于生成句子或段落的嵌入表示。这些嵌入可以用于各种下游任务,如语义相似性评估、文本分类等。以下是Sentence Transformers的一些关键特性:

\begin{itemize}
    \item \textbf{预训练模型:} Sentence Transformers利用了预训练的Transformer模型,这些模型已经在大量文本数据上进行了训练,能够捕捉到语言的深层语义信息。
    \item \textbf{可定制性:} 用户可以根据自己的需求选择不同的预训练模型,甚至可以微调模型参数以适应特定的应用场景。
    \item \textbf{生成嵌入表示:} 这些模型能够将句子或段落转换成固定长度的嵌入向量,这些向量可以用于后续的机器学习任务,如聚类、分类或最近邻搜索。
    \item \textbf{语义相似性评估:} Sentence Transformers在语义相似性评估任务中表现出色,它们能够量化句子之间的语义相似度或相关性。
    \item \textbf{跨语言能力:} 一些Sentence Transformers模型支持跨语言的文本比较,即使在不同语言之间也能评估文本的相似性。
    \item \textbf{应用场景:} 这些模型被广泛应用于问答系统、对话系统、文本摘要、法律文档分析等众多NLP领域。
\end{itemize}

Sentence Transformers在服务监管语言的实现中可以用于自动化地评估监管规则的语义相似性,帮助监管者识别和整合相似或冗余的规则,从而提高监管规则库的质量和效率。

\cleardoublepage

\section{服务监管语言设计}

本章介绍服务监管语言的详细设计,它通过结构化、领域无关的方法对规则进行建模,作为服务监管系统的基础。在服务监管系统中,本项目将该语言命名为\textbf{HORAE}(HOREA,希腊神话中的秩序女神),故后文都用\textbf{HORAE语言}指代。

\subsection{设计原则}

为了实现有效的HORAE语言,其设计采用了以下原则:

\begin{itemize}
    \item \textbf{通用性:} 一个与领域无关的语言应该能够适应不同领域和多语言文本的监管规则,这些规则在领域特定术语、模式和写作风格上可能存在显著差异。因此,它需要一个通用的范式,该范式抽象掉了领域特定的成分,同时仍能表达从各个领域提取的共通的规则模式。
    \item \textbf{结构化:} 尽管遵循通用性原则,但要设计的语言应保持(半)结构化,以便能够有效地存储、检查和操作监管规则,并尽可能解决潜在的歧义问题。
    \item \textbf{自动化:} 语言应具有简单的模式和结构,从而允许通过语法解析器或大型语言模型(LLMs)等手段,将自然语言形式的监管规则自动转换为建模语言。
    \item \textbf{可量化:} 由于下游识别模型和算法固有的不精确性,语言不仅应编码关于监管规则的定性信息(能否被满足),还应编码其定量方面(可满足的概率)。
\end{itemize}

\subsection{特性归纳及语法设计}

本研究对HORAE语言的语法设计遵循了归纳推理的范式:首先,本研究收集了50个领域的服务监管规则作为基准数据集,然后对这个基准数据集进行分析以提取语法的关键模式和特点:

\begin{itemize}
    \item \textbf{无依赖性}:无论表述的自然语言的语法结构如何,两个文本句子可能会编码语义上相似的监管规则。例如,“Employees must wash hands before returning to work”,“工作恢复前洗手是所有员工的必备条件”,“在返回工作前,员工必须洗手”这三条规则,尽管它们在自然语法上存在显著差异,但实际上代表了相同的监管意图。因此,HORAE语言的语法应不依赖于任何特定的自然语言语法,优化并接纳多样化的规则意图,减少语法类别的复杂性。
    \item \textbf{规则类型}:监管规则本质上是类型良好的,通常描述某些行为,这些行为可能被强制执行、被推荐或被禁止。例如,“员工必须在工厂地板上时始终戴安全眼镜”(强制执行),“建议所有参与者在操作任何机器之前查阅安全手册”(推荐),“禁止在距离加油泵50英尺内吸烟”(禁止)。因此,HORAE语言应提供简单的机制来指定规则的类型。
    \item \textbf{事件组成}:监管规则的事件描述具有高度的组成性。一个受监管的事件通常会通过逻辑连接词组合几个子事件。例如,“公司必须进行彻底的测试,并获得FDA批准或确保符合国际健康法规”。这种组合性是服务监管中的重要特性,因为它促成了将复杂的监管问题分解为一组可以更简单、更准确地解决的子问题。HORAE语言将通过维护基本事件的抽象层来支持组成性,这些事件编码了受规管实体的子事件,并可以逻辑地组合以描述整个事件。
    \item \textbf{时间性}:时间属性在服务监管中也是一个重要的特性,特别是在时间约束至关重要的应用领域,如金融服务,它们尤为突出。例如,“上市公司必须在季度结束后的45天内公开他们的季度财务结果;如果在这45天内发生了任何重大的金融事件,如合并或收购,那么必须在事件发生后的5天内提交额外的初步报告。”因此,HORAE语言应支持时间性,通过接纳带时间戳的事件和时间约束来实现。
\end{itemize}

根据上述观察,对HORAE中的规则进行建模,遵循以下抽象语法:

\begin{figure}[ht]
    \centering
    \includegraphics[width=.2\linewidth]{logo/zju}
\end{figure}

这种抽象语法由\textbf{顶层语法}和\textbf{底层语法}组成,虚线标示了它们之间的分界。顶层语法将(可能带有时间戳的)基本事件通过逻辑连接词组合成特定类型的规则,而底层语法则将更细粒的组成部分组装成基本事件。将基本事件细分为更小的、可检测的成分,可以提高下游识别模型和算法的精确度。

\begin{itemize}
    \item \textbf{顶层语法:} 在这一层,基本事件被视为最小的句法单位,它们将在形式语义中被解释为命题。该语法允许通过逻辑连接词组合基本事件,并指定所得到的规则的类型——“shall”、“should”和“forbid”,分别代表强制执行、推荐和禁止的行为。对于具有时间属性的规则,相应的基本事件可以与时间戳τ相关联,表示其发生的时间;此外,时间戳τ = {τ1, τ2, ...}上的时间约束被收集到C(τ)中,它作为规则中的一种特定形式的语句。
    \item \textbf{底层语法:} 在这一层,描述了从规则数据集中提取的基本事件的组成成分。关键成分包括:(i) action:基本事件的行为;(ii) object:动作的行为者或接受者——通常是一个可检测的目标;(iii) attribute:对象或动作的属性(由.运算符选择),如数量、颜色、长度等;以及(iv) attribute ⋄ value,其中 ⋄ ∈ {<, >, ≤, ≥, =}。某些属性与给定值(例如阈值、颜色等)的比较,这在服务监管中常用。
\end{itemize}

\subsection{语法分析器}

\subsubsection{基于Antlr语法分析器生成工具}

在HORAE语言的语法实现方面,本研究首先采用了Antlr(Another Tool for Language Recognition)作为语法分析器生成工具。Antlr是一款强大的语法分析器生成器,它能够自动构造词法分析器和语法分析器,极大地简化了编程语言或特定领域语言的解析过程。

Antlr的使用分为几个步骤:首先,定义语言的词法规则(Lexer)和语法规则(Parser),这些规则描述了语言的结构;然后,Antlr根据这些定义生成相应的分析器代码;最后,将生成的代码集成到应用程序中,以实现对语言的解析。

对于HORAE语言,先详细定义了其词法单元和语法结构,包括基本事件、规则类型、时间戳和逻辑连接词等。接着,利用Antlr4工具,生成了HORAE语言的解析器。该解析器能够接收HORAE语言的文本输入,生成对应的抽象语法树(AST),进而为后续的形式语义分析和自动化处理提供了基础。

此外,Antlr提供的树形结构操作库(如CommonTree和CommonTreeNodeStream)也在后续工作中发挥了作用。通过这些库,本项目可以方便地遍历和处理抽象语法树,以支持下一步的应用。

以一条具体的规则为例:

\textbf{道路两侧原有建筑或小区新增违法建筑物,未经部门审批就产权区域面积超过100m²}

进行词法分析,生成如下token流:

\textbf{((object action object)(object action object))(!object action)(object attribute>value)}

再进行语法分析,生成如下抽象语法树:

通过Antlr工具,HORAE语言的语法分析变得更加系统化和自动化,提高了语言处理的效率和准确性。

\subsubsection{基于大语言模型的语法分析器}

随着项目的深入,本研究逐渐意识到在多样化的数据集面前,Antlr的解析能力存在一定的局限性。Antlr通常适用于构建那些规范性较强的编程语言的语法分析器,然而HORAE语言面对的规则内容往往更贴近自然语言的范畴。对于城市管理这类数据集,由于其语法结构相对简单,HORAE语言还能够设计出符合Antlr规范的语法规则。但是,当涉及更广泛的领域和更复杂的基础事件内部结构时,继续遵循Antlr的严格规范来设计语法就变得愈发具有挑战性。

鉴于此,HORAE语言决定保留已有的语法设计框架,同时采用大语言模型进行语法解析,不再受限于Antlr的具体规范。这种方法允许HORAE利用LLMs在自然语言处理方面的强大能力,以更加灵活和适应性强的方式处理各种复杂的规则内容。

大语言模型通过在大量文本数据上的预训练,已经具备了对自然语言的深入理解能力,这使得它们能够在识别和解析非规范性语言结构方面表现出色。通过微调这些模型,HORAE语言能够进一步提高其对特定领域规则内容的解析精度,从而有效地提升HORAE语法分析器的性能。

此外,使用大语言模型还为HORAE语言带来了增量式处理的能力。与传统编程语言的语法分析器不同,后者在扩展或修改语法时通常需要从头开始重建。得益于RuleGPT在微调时使用LoRA(低阶适应)技术能够保留先前知识的能力,大语言模型可以仅通过使用额外的数据集对LLMs 进行微调来适应新的事件模式,而无需对整个基准数据集重新训练。这一特性极大地提高了HORAE语言的灵活性和可维护性。

使用大语言模型进行语法分析的实现细节,见第5章自动化。

\cleardoublepage

\section{HORAE语义建模}


\cleardoublepage

\section{自动化}


















\par 我们可以用includegraphics来插入现有的jpg等格式的图片,如\autoref{fig:zju-logo}。

\begin{figure}[ht]
    \centering
    \includegraphics[width=.4\linewidth]{logo/zju}
    \caption{\label{fig:zju-logo}浙江大学LOGO}
\end{figure}

\par 如\autoref{tab:sample}所示,这是一张自动调节列宽的表格。

\begin{table}[ht]
    \caption{\label{tab:sample}自动调节列宽的表格}
    \begin{tabularx}{\linewidth}{|c|X<{\centering}|}
        \hline
        第一列 & 第二列 \\ \hline
        xxx & xxx \\ \hline
        xxx & xxx \\ \hline
        xxx & xxx \\ \hline
    \end{tabularx}
\end{table}

\par 如\autoref{equ:sample},这是一个公式

\begin{equation}
    \label{equ:sample}
    A=\overbrace{(a+b+c)+\underbrace{i(d+e+f)}_{\text{虚数}}}^{\text{复数}}
\end{equation}

\par 如\autoref{code:sample}所示,这是一段代码。
计算机学院的代码样式可能与其他专业不同,
如有需要,可以从计算机学院专业模板中复制相关的代码样式设定。

\begin{lstlisting}[%
    language={C},
    caption={simple.c},
    label={code:sample}
]
#include <stdio.h>

int main(int argc, char *argv[])
{
    printf("Hello, zjuthesis\n");
    return 0;
}
\end{lstlisting}

\subsection{关于字体}

英文字体通常提供了粗体和斜体的组合,中文字体通常没有粗体或斜体,本模板使用了 `AutoFakeBold' 来实现中文伪粗体,但不提供中文斜体,如\autoref{tab:font-examples}所示。

\begin{table}
    \centering
    \caption{一些字体示例}
    \label{tab:font-examples}
    \begin{tabular}{|c|c|c|c|c|}
        \hline
        字体            & 常规             & 粗体                       & 斜体                      & 粗斜体                                \\ \hline
        Times New Roman & Regular         & {\bfseries          Bold} & {\itshape         Italic} & {\bfseries \itshape      BoldItalic} \\ \hline
        仿宋            & {\fangsong 常规} & {\fangsong \bfseries 粗体} & {\fangsong \itshape 斜体} & {\fangsong \bfseries \itshape 粗斜体} \\ \hline
        宋体            & {\songti   常规} & {\songti   \bfseries 粗体} & {\songti   \itshape 斜体} & {\songti   \bfseries \itshape 粗斜体} \\ \hline
        黑体            & {\heiti    常规} & {\heiti    \bfseries 粗体} & {\heiti    \itshape 斜体} & {\heiti    \bfseries \itshape 粗斜体} \\ \hline
        楷体            & {\kaishu   常规} & {\kaishu   \bfseries 粗体} & {\kaishu   \itshape 斜体} & {\kaishu   \bfseries \itshape 粗斜体} \\ \hline
    \end{tabular}
\end{table}

\sectionnonum[none]{同一页上的章标题}
