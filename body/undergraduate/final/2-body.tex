\cleardoublepage

\section{背景知识与技术}

\subsection{服务监管系统架构}

服务监管语言是服务监管系统的重要组成部分,下图即为服务监管系统的整体架构。从多模态规则数据集开始,经过服务监管语言处理,形成半结构化的规则数据库,然后按照语法规范经过进一步处理提取规则的更细粒的层次结构形成语法树,再进行语义建模和语义检测,然后应用于大模型的服务监管工作中。

\subsection{Antlr语法分析器生成工具}

Antlr,全称为“ANother Tool for Language Recognition”,是一款功能强大的语法分析器生成工具,主要用于从语言规范中读取构造语法分析器或者转译器的代码。

Antlr具有以下几个主要特点:

\begin{itemize}
    \item 易用性:Antlr提供了一种简洁、直观的语法表示方式,使得用户能够快速地定义和理解语法规则。
    \item 强大的解析能力:Antlr能够处理广泛的语法结构,包括LL(*)和LR(k)等。无论是处理简单的文本格式,还是复杂的编程语言,Antlr都能够应对自如。
    \item 丰富的目标语言支持:Antlr支持生成多种编程语言的解析器,包括Java、C++、Python、JavaScript等,满足了用户的多样化需求。
    \item 强大的错误处理机制:Antlr具有强大的错误处理和恢复机制,使得解析器能够在出错时提供有用的反馈,帮助用户识别和修复问题。
    \item 易于集成:Antlr生成的解析器易于集成到用户的应用程序中,无论是作为一个独立的组件,还是作为一个库,都能够轻松实现。
\end{itemize}

综上,Antlr是一款功能强大、易用的语法分析器生成工具,无论是用于编译原理的学习,还是用于实际的语言设计和实现,都是一个理想的选择。

\subsection{节标题}

\subsubsection{小节标题}

\par 我们可以用includegraphics来插入现有的jpg等格式的图片,如\autoref{fig:zju-logo}。

\begin{figure}[ht]
    \centering
    \includegraphics[width=.4\linewidth]{logo/zju}
    \caption{\label{fig:zju-logo}浙江大学LOGO}
\end{figure}

\par 如\autoref{tab:sample}所示,这是一张自动调节列宽的表格。

\begin{table}[ht]
    \caption{\label{tab:sample}自动调节列宽的表格}
    \begin{tabularx}{\linewidth}{|c|X<{\centering}|}
        \hline
        第一列 & 第二列 \\ \hline
        xxx & xxx \\ \hline
        xxx & xxx \\ \hline
        xxx & xxx \\ \hline
    \end{tabularx}
\end{table}

\par 如\autoref{equ:sample},这是一个公式

\begin{equation}
    \label{equ:sample}
    A=\overbrace{(a+b+c)+\underbrace{i(d+e+f)}_{\text{虚数}}}^{\text{复数}}
\end{equation}

\par 如\autoref{code:sample}所示,这是一段代码。
计算机学院的代码样式可能与其他专业不同,
如有需要,可以从计算机学院专业模板中复制相关的代码样式设定。

\begin{lstlisting}[%
    language={C},
    caption={simple.c},
    label={code:sample}
]
#include <stdio.h>

int main(int argc, char *argv[])
{
    printf("Hello, zjuthesis\n");
    return 0;
}
\end{lstlisting}

\subsection{关于字体}

英文字体通常提供了粗体和斜体的组合,中文字体通常没有粗体或斜体,本模板使用了 `AutoFakeBold' 来实现中文伪粗体,但不提供中文斜体,如\autoref{tab:font-examples}所示。

\begin{table}
    \centering
    \caption{一些字体示例}
    \label{tab:font-examples}
    \begin{tabular}{|c|c|c|c|c|}
        \hline
        字体            & 常规             & 粗体                       & 斜体                      & 粗斜体                                \\ \hline
        Times New Roman & Regular         & {\bfseries          Bold} & {\itshape         Italic} & {\bfseries \itshape      BoldItalic} \\ \hline
        仿宋            & {\fangsong 常规} & {\fangsong \bfseries 粗体} & {\fangsong \itshape 斜体} & {\fangsong \bfseries \itshape 粗斜体} \\ \hline
        宋体            & {\songti   常规} & {\songti   \bfseries 粗体} & {\songti   \itshape 斜体} & {\songti   \bfseries \itshape 粗斜体} \\ \hline
        黑体            & {\heiti    常规} & {\heiti    \bfseries 粗体} & {\heiti    \itshape 斜体} & {\heiti    \bfseries \itshape 粗斜体} \\ \hline
        楷体            & {\kaishu   常规} & {\kaishu   \bfseries 粗体} & {\kaishu   \itshape 斜体} & {\kaishu   \bfseries \itshape 粗斜体} \\ \hline
    \end{tabular}
\end{table}

\sectionnonum[none]{同一页上的章标题}
