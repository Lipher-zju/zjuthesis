\cleardoublepage

\section{背景知识与技术}

在探讨服务监管语言的设计之前,有必要了解一些关键的背景知识和技术,这些将为后续的研究提供理论和实践基础。

\subsection{服务监管系统架构}

\begin{figure}[ht]
    \centering
    \includegraphics[width=.4\linewidth]{logo/zju}
    \caption{\label{fig:total-struct}服务监管系统架构}
\end{figure}

服务监管语言是构建高效服务监管系统的核心要素,它在整个监管框架中扮演着至关重要的角色。\autoref{fig:total-struct}描述了服务监管系统的工作流程和架构概览,具体包括:

\begin{itemize}
    \item \textbf{数据集的整合与预处理:} 系统从多模态的规则数据集着手,这些数据集可能包含文本、语音、图像等多种格式的规则信息。
    \item \textbf{规则数据库的初步构建:} 通过应用服务监管语言,原始的多模态规则数据被转换成一种半结构化的格式,进而形成规则数据库。这一步骤是将非结构化数据转换为机器可读格式的关键环节。
    \item \textbf{语法规范与处理:} 遵守既定的语法规范,系统对半结构化的规则数据进行进一步的精细化处理,提炼出规则的深层次结构,并构建成语法树。语法树的构建为后续的语义分析和规则执行提供了清晰的逻辑框架。
    \item \textbf{语义建模与检测:} 在语法树的基础上,系统进行语义建模,将规则中的自然语言转化为机器可理解的语义表达。同时,系统执行语义检测,确保规则的一致性和无歧义性,避免潜在的逻辑矛盾。
    \item \textbf{大模型服务监管的实现:} 最后,经过上述步骤处理的规则被应用于大模型驱动的服务监管工作中。大模型凭借其强大的数据处理和智能识别能力,能够高效地执行监管任务,实现服务的自动化和智能化监管。
\end{itemize}

整个服务监管系统的架构设计旨在实现规则的高效管理、精准执行和智能监管。服务监管语言的设计和应用,确保了规则从原始数据到最终执行的每一步都准确无误,为服务监管的自动化和智能化提供了坚实的基础。通过这种系统化的方法,我们能够应对日益增长的监管需求,提高监管的效率和效果,确保服务的合规性和安全性。

\subsection{Antlr语法分析器生成工具}

Antlr,全称为“ANother Tool for Language Recognition”,是一款功能强大的语法分析器生成工具,主要用于从语言规范中读取构造语法分析器或者转译器的代码。它有以下几个主要特点:

\begin{itemize}
    \item \textbf{易用性:} Antlr提供了一种简洁、直观的语法表示方式,使得用户能够快速地定义和理解语法规则。
    \item \textbf{强大的解析能力:} Antlr能够处理广泛的语法结构,包括LL(*)和LR(k)等。无论是处理简单的文本格式,还是复杂的编程语言,Antlr都能够应对自如。
    \item \textbf{丰富的目标语言支持:} Antlr支持生成多种编程语言的解析器,包括Java、C++、Python、JavaScript等,满足了用户的多样化需求。
    \item \textbf{强大的错误处理机制:} Antlr具有强大的错误处理和恢复机制,使得解析器能够在出错时提供有用的反馈,帮助用户识别和修复问题。
    \item \textbf{易于集成:} Antlr生成的解析器易于集成到用户的应用程序中,无论是作为一个独立的组件,还是作为一个库,都能够轻松实现。
\end{itemize}

Antlr是一款功能强大、易用的语法分析器生成工具,无论是用于编译原理的学习,还是用于实际的语言设计和实现,都是一个理想的选择。

\subsection{大语言模型}
大语言模型(Large Language Models,LLMs)在自然语言处理(NLP)领域中是一个重大的突破。它们代表了深度学习在模拟人类语言理解方面所取得的进展。以下是大语言模型的一些核心特性:

\begin{itemize}
    \item \textbf{规模与复杂性:} 大语言模型利用大量的参数(通常是数十亿到数千亿的数量级)来捕获和理解语言的复杂性。这种规模的模型能够学习到语言的深层结构和细微差别,从而在各种语言任务中展现出优异的性能。
    \item \textbf{预训练与微调:} 这些模型通常首先在大规模的数据集上进行预训练,学习语言的通用表示。预训练阶段使模型能够积累广泛的知识,之后在特定任务上的微调则进一步提升了模型的专业性,优化其性能。
    \item \textbf{上下文理解:} 大语言模型能够理解文本的上下文信息,这使得它们能够生成连贯且富有深度的文本内容。这种上下文感知能力是实现高级语言任务,如理解隐喻、同义等复杂结构的关键。
    \item \textbf{多任务学习:} 它们具有处理多种不同任务的能力,如文本分类、问答、文本生成等。这种多任务学习的能力使得大语言模型可以灵活地应用于广泛的应用场景,显示出了令人印象深刻的适应性和灵活性。
    \item \textbf{生成能力:} 大语言模型不仅能够理解语言,还能够生成语言。这使得它们在创意写作、内容生成和对话系统等领域具有广泛的应用潜力。
\end{itemize}

大语言模型在服务监管语言的设计和实施中发挥了关键作用。它们可以用于增强语言的自动解析能力,提升对语义的理解,并执行复杂的逻辑推理任务。此外,大模型还能够辅助监管者在制定和执行规则时更好地理解和应用语言的细微差别,从而提高服务监管的准确性和效率。

\subsection{随机布尔可满足性}
随机布尔可满足性(Stochastic Boolean Satisfiability, SSAT)是逻辑和计算机科学中的一个概念,它在处理不确定性和概率性问题时非常有用。SSAT是传统布尔可满足性问题(Boolean Satisfiability Problem, SAT)的扩展,其中引入了概率因素。以下是SSAT的一些核心概念和应用:

\begin{itemize}
    \item \textbf{概率模型:} 在SSAT中,每个布尔变量被赋予一个满足概率,即该变量为真的可能性。这种概率化表示使得SSAT能够模拟现实世界中的不确定性。
    \item \textbf{逻辑约束:} 与传统的SAT问题类似,SSAT中的公式由布尔逻辑运算符(如AND、OR和NOT)连接的变量组成。不同之处在于,SSAT考虑了每个变量的满足概率。
    \item \textbf{可满足性计算:} SSAT的目标是找到一组变量赋值,使得整个公式的满足概率最大化,或者计算在给定约束下公式满足的概率。
    \item \textbf{应用领域:} SSAT在多个领域都有应用,包括机器学习、人工智能、风险评估和决策支持系统。它特别适用于需要在不确定性下做出决策的场景。
    \item \textbf{算法和求解器:} 解决SSAT问题通常需要专门的算法和求解器。这些工具可以处理复杂的逻辑公式,并计算出满足特定概率阈值的解决方案。
    \item \textbf{与服务监管的联系:} 在服务监管语言的设计中,SSAT可以用来评估监管规则在不同情况下的满足概率,帮助监管者理解规则的潜在影响,并优化规则以提高监管效率。
\end{itemize}

SSAT作为一种强大的工具,为服务监管语言提供了一种评估和优化监管规则的新方法。通过考虑概率因素,监管者可以更全面地理解和应对监管环境中的不确定性。

\subsection{Z3求解器}
Z3是一个由Microsoft Research开发的高效SMT(Satisfiability Modulo Theories)求解器,它在形式化验证、程序验证和符号计算等多个领域有着广泛的应用。以下是Z3求解器的一些关键特性:

\begin{itemize}
    \item \textbf{多种逻辑支持:} Z3支持多种逻辑,包括但不限于二阶逻辑、量化逻辑和各种片段的一阶逻辑,使其能够解决广泛的数学和逻辑问题。
    \item \textbf{高性能:} Z3采用了许多先进的优化技术,如按需模型生成、高效的搜索策略和并行处理,以提高求解速度和效率。
    \item \textbf{可扩展性:} 设计上注重可扩展性,Z3能够处理大规模的问题实例,同时保持相对合理的性能。
    \item \textbf{集成与API:} 提供了丰富的API,支持多种编程语言,包括C++、Python、Scala等,方便开发者将其集成到不同的应用中。
    \item \textbf{应用场景:} Z3被用于软件和硬件验证、网络安全、优化问题、机器学习模型的验证以及形式化方法教育等领域。
    \item \textbf{定理证明:} 在定理证明中,Z3可以自动化地证明给定的公式或者查找使其为假的反例,大大简化了证明过程。
    \item \textbf{挑战与限制:} 尽管Z3非常强大,但它也有局限性,如处理某些特定类型问题时的性能瓶颈,以及对于非专业人员来说较高的使用门槛。
\end{itemize}

Z3求解器在服务监管语言的实现中扮演了重要角色,特别是在需要形式化验证和逻辑推理的场景中。通过使用Z3,开发者可以确保监管规则的逻辑正确性,并自动化地验证规则之间的一致性和完整性。

\subsection{Sentence Transformers}
Sentence Transformers是由德国柏林工业大学的团队开发的,它们基于预训练的语言模型,如BERT或RoBERTa,用于生成句子或段落的嵌入表示。这些嵌入可以用于各种下游任务,如语义相似性评估、文本分类等。以下是Sentence Transformers的一些关键特性:

\begin{itemize}
    \item \textbf{预训练模型:} Sentence Transformers利用了预训练的Transformer模型,这些模型已经在大量文本数据上进行了训练,能够捕捉到语言的深层语义信息。
    \item \textbf{可定制性:} 用户可以根据自己的需求选择不同的预训练模型,甚至可以微调模型参数以适应特定的应用场景。
    \item \textbf{生成嵌入表示:} 这些模型能够将句子或段落转换成固定长度的嵌入向量,这些向量可以用于后续的机器学习任务,如聚类、分类或最近邻搜索。
    \item \textbf{语义相似性评估:} Sentence Transformers在语义相似性评估任务中表现出色,它们能够量化句子之间的语义相似度或相关性。
    \item \textbf{跨语言能力:} 一些Sentence Transformers模型支持跨语言的文本比较,即使在不同语言之间也能评估文本的相似性。
    \item \textbf{应用场景:} 这些模型被广泛应用于问答系统、对话系统、文本摘要、法律文档分析等众多NLP领域。
\end{itemize}

Sentence Transformers在服务监管语言的实现中可以用于自动化地评估监管规则的语义相似性,帮助监管者识别和整合相似或冗余的规则,从而提高监管规则库的质量和效率。

\cleardoublepage

\section{服务监管语言设计}

本章将详细介绍服务监管语言的设计过程。该语言采用结构化且与领域无关的方法对规则进行建模,从而构成了服务监管系统的核心基础。为了体现其在维持系统秩序中的关键作用,本项目将该语言命名为 \textbf{HORAE}\footnote{源自希腊神话中的秩序女神 \textit{Horeā}}。在后续的文档中,我们将统一使用 \textbf{HORAE }  来指代该服务监管语言。

\subsection{设计原则}

为了实现有效的HORAE ,其设计采用了以下原则:

\begin{itemize}
    \item \textbf{通用性:} HORAE 必须是领域无关的,能够适应不同领域以及多语言环境下的监管规则。这些规则在特定领域的术语、模式和写作风格上可能存在显著差异。因此,我们需要一个通用的设计范式,它能够抽象出领域特定的元素,同时能够表达跨领域通用的规则模式。
    \item \textbf{结构化:} 在确保 HORAE 的通用性的同时,也需要保证其(半)结构化特性。这样的设计可以使得监管规则能够有效地被存储、检查和操作,并且尽可能地减少潜在的歧义,从而提高规则的可理解性和可操作性。
    \item \textbf{自动化:} HORAE 的设计注重简化模式和结构,以便可以利用语法解析器或大型语言模型(LLMs)等工具,实现从自然语言表述的监管规则到形式化建模语言的自动转换。这一自动化过程将极大地提升规则处理的效率和准确性。
    \item \textbf{可量化:} 考虑到监管规则的执行可能涉及不确定性,HORAE语言不仅应编码监管规则的定性信息(即规则是否可被满足),还应包含定量信息(即规则满足的概率)。这种量化的信息对于评估和优化监管策略至关重要。
\end{itemize}

\subsection{特性归纳及语法设计}

本研究对 HORAE 的语法设计遵循了归纳推理的范式:首先,本研究收集了50个领域的服务监管规则作为基准数据集,然后对这个基准数据集进行分析以提取语法的关键模式和特点:

\begin{itemize}
    \item \textbf{无依赖性}:无论表述的自然语言的语法结构如何,两个文本句子可能会编码语义上相似的监管规则。例如,“Employees must wash hands before returning to work”,“工作恢复前洗手是所有员工的必备条件”,“在返回工作前,员工必须洗手”这三条规则,尽管它们在自然语法上存在显著差异,但实际上代表了相同的监管意图。因此,HORAE 的语法应不依赖于任何特定的自然语言语法,优化并接纳多样化的规则意图,减少语法类别的复杂性。
    \item \textbf{规则类型}:监管规则本质上是类型良好的,通常描述某些行为,这些行为可能被强制执行、被推荐或被禁止。例如,“员工必须在工厂地板上时始终戴安全眼镜”(强制执行),“建议所有参与者在操作任何机器之前查阅安全手册”(推荐),“禁止在距离加油泵50英尺内吸烟”(禁止)。因此,HORAE 应提供简单的机制来指定规则的类型。
    \item \textbf{事件组成}:监管规则的事件描述具有高度的组成性。一个受监管的事件通常会通过逻辑连接词组合几个子事件。例如,“公司必须进行彻底的测试,并获得FDA批准或确保符合国际健康法规”。这种组合性是服务监管中的重要特性,因为它促成了将复杂的监管问题分解为一组可以更简单、更准确地解决的子问题。HORAE 将通过维护基本事件的抽象层来支持组成性,这些事件编码了受规管实体的子事件,并可以逻辑地组合以描述整个事件。
    \item \textbf{时间性}:时间属性在服务监管中也是一个重要的特性,特别是在时间约束至关重要的应用领域,如金融服务,它们尤为突出。例如,“上市公司必须在季度结束后的45天内公开他们的季度财务结果;如果在这45天内发生了任何重大的金融事件,如合并或收购,那么必须在事件发生后的5天内提交额外的初步报告。”因此,HORAE 应支持时间性,通过接纳带时间戳的事件和时间约束来实现。
\end{itemize}

根据上述观察,进行了 HORAE 的语法建模,形成了一套抽象语法。这套语法由两个核心层次构成。通过顶层语法,我们可以将复杂的监管情境转化为清晰、结构化的规则。底层语法则关注于构建基本事件的更细微的组成部分,从而为识别模型和算法提供更精确的输入数据。
\begin{align*}
    R \ \  \Coloneqq \ \  & type~s \tag{typed rule}\\
    type \ \  \Coloneqq \ \  & shall~|~should~|~forbid \tag{predefined types}\\
    s \ \  \Coloneqq \ \  &  \neg s~|~s~ \land~ s ~|~s~ \lor~ s ~|~
    \langle~\tau,~e \rangle~|~e~|\mathcal{C}(\tau) \tag{statement}\\
    \cdot \cdot \cdot \cdot\cdot\cdot\cdot\cdot\cdot\cdot & \cdot\cdot\cdot\cdot\cdot\cdot\cdot\cdot\cdot\cdot\cdot\cdot\cdot\cdot\cdot\cdot\cdot\cdot \cdot \cdot \cdot \cdot \cdot \cdot \cdot \cdot\cdot\\
    e \ \  \Coloneqq \ \  & object~action~| \tag{patterned event}\\
    \Coloneqq \ \  & object~action~object~|\\
    \Coloneqq \ \  & action~object~|\\
    \Coloneqq \ \  & object. attribute \diamond value|\\
    \Coloneqq \ \  & action. attribute \diamond value
\end{align*}
虚线区分了顶层语法与底层语法。通过这种分层的语法设计,HORAE 能够以一种高度结构化和精确的方式,表达服务监管的复杂规则。



\begin{itemize}
    \item \textbf{顶层语法:} 在这一层,基本事件被视为最小的句法单位,它们将在形式语义中被解释为命题。该语法允许通过逻辑连接词组合基本事件,并指定所得到的规则的类型——“shall”、“should”和“forbid”,分别代表强制执行、推荐和禁止的行为。对于具有时间属性的规则,相应的基本事件可以与时间戳$\tau$相关联,表示其发生的时间;此外,时间戳$\tau$ = {$\tau1$, $\tau2$, ...}上的时间约束被收集到C($\tau$ )中,它作为规则中的一种特定形式的基本事件。
    \item \textbf{底层语法:} 在这一层,描述了从规则数据集中提取的基本事件的组成成分。关键成分包括:(i) action:基本事件的行为;(ii) object:动作的行为者或接受者——通常是一个可检测的目标;(iii) attribute:对象或动作的属性(由.运算符选择),如数量、颜色、长度等;以及(iv) attribute $\diamond$ value,其中$\diamond \in \{<, >, \leq, \geq, =\}$。某些属性与给定值(例如阈值、颜色等)的比较,这在服务监管中常用。
\end{itemize}

\subsection{语法分析器}

\subsubsection{基于Antlr语法分析器生成工具}

在 HORAE 的语法实现方面,本研究首先采用了Antlr(Another Tool for Language Recognition)作为语法分析器生成工具。Antlr是一款强大的语法分析器生成器,它能够自动构造词法分析器和语法分析器,极大地简化了编程语言或特定领域语言的解析过程。

Antlr的使用分为几个步骤:首先,定义语言的词法规则(Lexer)和语法规则(Parser),这些规则描述了语言的结构;然后,Antlr根据这些定义生成相应的分析器代码;最后,将生成的代码集成到应用程序中,以实现对语言的解析。

对于 HORAE ,先详细定义了其词法单元和语法结构,包括基本事件、规则类型、时间戳和逻辑连接词等。接着,利用Antlr4工具,生成了 HORAE 的解析器。该解析器能够接收 HORAE 的文本输入,生成对应的抽象语法树(AST),进而为后续的形式语义分析和自动化处理提供了基础。

此外,Antlr提供的树形结构操作库(如CommonTree和CommonTreeNodeStream)也在后续工作中发挥了作用。通过这些库,本项目可以方便地遍历和处理抽象语法树,以支持下一步的应用。以一条具体的规则为例:

\textbf{道路两侧原有建筑或小区新增违法建筑物,未经部门审批就产权区域面积超过100m²}

进行词法分析,生成如下token流:

\textbf{object action object $\lor$ object action object $\lor$ !object action $\land$ object attribute > value}


再进行语法分析,生成如下抽象语法树:

\begin{figure}[ht]
    \centering
    \includegraphics[width=.3\linewidth]{logo/zju}
\end{figure}

通过Antlr工具,HORAE 的语法分析变得更加系统化和自动化,提高了语言处理的效率和准确性。

\subsubsection{基于大语言模型的语法分析器}

随着研究的不断深入,我们逐渐认识到,在多样化的数据集面前,Antlr 的解析能力存在一定的局限性。Antlr 擅长构建规范性较强的编程语言的语法分析器,而 HORAE 所面临的规则内容,往往更接近自然语言的复杂性。对于城市管理等语法结构相对简单的数据集,HORAE 尚能设计出符合 Antlr 规范的语法规则。然而,面对更广泛的领域和更复杂的基础事件内部结构,继续遵循 Antlr 的严格规范设计语法变得日益困难。

因此,HORAE 决定保留现有的语法设计框架,同时引入大语言模型(Large Language Models, LLMs)进行语法解析,从而突破 Antlr 规范的局限。这种方法使得 HORAE 能够利用 LLMs 在自然语言处理方面的强大能力,以更灵活、更具适应性的方式处理各种复杂的规则内容。

通过在大量文本数据上的预训练,大语言模型已经展现出对自然语言的深入理解,特别是在识别和解析非规范性语言结构方面。通过对这些模型进行微调,HORAE 能够显著提升对特定领域规则内容的解析精度,进而有效提高语法分析器的整体性能。

此外,采用大语言模型还为 HORAE 带来了增量式处理的能力。与需要从头重建的传统编程语言语法分析器不同,大语言模型可以通过 RuleGPT 使用 LoRA(低阶适应)技术进行微调,保留先前知识,仅通过额外的数据集对 LLMs 进行微调,以适应新的事件模式,无需对整个基准数据集重新训练。这一特性极大地增强了 HORAE 的灵活性和可维护性。

有关使用大语言模型进行语法分析的实现细节,请参见第5章 \textbf{自动化}。

\cleardoublepage

\section{语义设计}

HORAE 的语义致力于清晰明确地阐释符合 HORAE 语法的规则所表达的含义。这种语义对于理解和推理通常数量庞大的规则集至关重要。它不仅为监管规则的表示、解释和计算提供了基础,还特别提供了检查规则库的一致性的机制,以便在实际应用到监管任务之前发现并解决潜在的冲突。通过这种方式,可以确保规则库在部署时的可靠性和有效性,避免因规则间的矛盾导致的监管失误。

\subsection{语义建模}

语法设计完成后,规则语言的语义变得清晰:基于语法,逻辑上组合基本事件的可满足性概率。进行语法分析生成解析树后,树的层次结构代表着基本事件之间的逻辑关系。这种逻辑关系可以等同于逻辑表达式。这种模式与SSAT(随机可满足性)模型非常吻合;每个基本事件可以被视为SSAT中的一个布尔随机变量。每个基本事件都有自己的可满足性概率,基本事件之间的逻辑关系直接对应于SSAT中的逻辑运算。因此,我们以量化的SSAT形式构建了规则语言的语义。

例如,考虑规则 $R1=$ “如果乘客没有房间或者房间干净且湿度标准,则...”,经过语法分析,我们可以将其分解为 $e1=$ “乘客有房间”,$e2=$ “房间干净”,$e3=$ “房间湿度标准”。然后这个规则的SSAT形式的语义表达可朴素的写为 $R1=\neg e1 \vee e2 \wedge e3$,其语义将是 $P1=1-P(e1)+P(e2)\times P(e3)$。

每条规则是基本事件的逻辑组合,一旦构建了规则库,它就成为一组命题公式。每个基本事件可以被视为一个命题变量(布尔变量),表示为 $\{e_1, \dots, e_k\}$。因此,规则库是包含这些基本事件的公式集合,表示为 $\{r_1, \dots, r_n\}$。设公式为
\begin{align*}
\varphi = \bigwedge_{i=1}^{n} r_i.
\end{align*}

一旦下游服务监管系统完成所有基本事件的系统检测,我们获得了一些基本事件子集 $\{e_{i_1}, \dots, e_{i_m}\} \subset \{e_1, ..., e_k\}$ 的赋值集合,表示为 $v: \{e_{i_1}, \dots, e_{i_m}\} \to \{0,1\}$。考虑公式
\begin{align*}
\exists e_{i_{m+1}} \dots \exists e_{i_{k}} \varphi',
\end{align*}
其中 $\{e_{i_{m+1}},\dots,e_{i_{k}}\}=\{e_1,\dots,e_k\} - \{e_{i_1}, \dots, e_{i_m}\}$,且 $\varphi'$ 是用 $v(e_{i_j})$ 替换公式 $\varphi$ 中出现的变量 $e_{i_j}$ 后的结果。如果给定的公式不可满足,则输入数据违反了规则库。这是一个典型的SAT问题,其语义是:给定一组输入,它们是否满足规则库中的所有规则。

然而,在现实世界的场景中,我们的检测设备通常无法准确判断某些基本事件是否满足;它们只能提供事件为真的相应概率。在这种情况下,对于每个命题变量,都有一个相关的概率值,表示其为真的可能性。这种赋值称为“随机赋值”。对服务监管系统给定一组输入数据(可能是多模态的:文本、图像或视频),我们可以为基本事件 $\{e_{i_1}, \dots, e_{i_m}\} \subset \{e_1, ..., e_k\}$ 获得一组随机赋值。设 $p_{i_j}$ 表示基本事件 $e_{i_j}$ 为真的概率。然后,公式
\begin{align*}
    \rotatebox{180}{R}^{p_{i_1}} e_{i_1} \dots \rotatebox{180}{R}^{p_{i_1}} e_{i_m}
    \exists e_{i_{m+1}} \dots \exists e_{i_k}
    \varphi
\end{align*}
具有以下语义:给定输入数据,满足规则库的最大概率。与确定性逻辑公式相比,这种描述更适合现实生活场景,因为多模态输入无法完全准确。编码后,这类问题的模式与SSAT(随机布尔可满足性)问题一致。因此,我们可以使用SSAT求解器来解决这个问题。

\subsection{相似性检验}

在根据 HORAE 解析规则时,基本事件相关性仍然是一个挑战:来自同一规则库 的基本事件在自然语言方面可能彼此语义相关或相同,尤其是跨不同规则语句的基本事件。那么如何将这些用自然语言描述的事件抽象成一组符号命题,同时保证语义相关性呢?实际上,这可以通过在自然语言处理中用于评估文本相似性的多种技术来实现,例如 TF-IDF、Word2Vec,以及 BERT。在一系列的测试中,Sentence Transformers 表现出了最佳性能。

\begin{figure}[ht]
    \centering
    \includegraphics[width=.3\linewidth]{logo/zju}
    \caption{相似性检验的情境,可知$e_{11} = e_{22} = e_{32}$,$e_{12} = \neg e_{33}$,$e_{13} = \neg e_{23}$,$e_{21} = \neg e_{31}$,$e_{14} = \neg e_{34}$}
\end{figure}

以下是利用Sentence Transformers 进行相似性检验的步骤:
\begin{itemize}
    \item \textbf{数据预处理:} 首先,将规则库中的所有基本事件提取出来,并进行必要的文本清洗,包括去除停用词、标点符号,以及执行词干提取或词形还原。
    \item \textbf{向量化表示:} 然后,使用Sentence Transformers 将每个清洗后的基本事件转换为固定长度的向量表示。这些向量能够捕捉事件之间的语义相似性。
    \item \textbf{相似性度量:} 接下来,计算每一对基本事件向量之间的相似度。这可以通过余弦相似度、欧氏距离或其他相似性度量方法来实现。
    \item \textbf{相似性阈值设定:} 为了确定两个事件是否足够相似,我们设定一个相似性阈值。如果两个事件的相似度超过这个阈值,则认为它们是一致的。需要注意的是阈值的设置需要非常保守,我们能够接受的本质一致的基本事件被认为不同,这只会导致性能的下降;但本质不同的基础事件被认为是一致的将会导致错误的出现。
\end{itemize}

通过这种方法,我们能够有效地处理来自同一规则库的基本事件之间的语义相关性,从而提高规则库的一致性和可管理性。此外,Sentence Transformers 的使用还为我们提供了一种灵活的方式来适应不同的规则库和领域,因为它们可以被训练或微调以适应特定的语料库或领域术语。

Sentence Transformers 的性能会受到训练数据质量和模型参数的影响。因此,实际应用需进行充分的测试和调整,以确保获得最佳的相似性检验结果。

\subsection{一致性检验}

一致性验证的目的是确定给定的规则库是否包含相互冲突的多个规则。当规则库中的所有规则都能同时被满足时,该规则库就是一致的。验证规则库一致性的核心思想是将其转化为一个可满足性问题(SAT)。首先,我们将基本事件视为布尔变量,表示为 $e_1, e_2, \ldots, e_k$。然后,我们可以将由基本事件组成的规则转换为布尔表达式,表示为 $r_1, r_2, \ldots, r_n$。对于给定的规则库,其所有规则可以形成如下的合取范式(CNF):
$$CNF=\bigwedge_{i=1}^{n}r_i=r_1 \land r_2 \land \cdots \land r_n$$

Z3求解器可以用来证明这个 CNF 是否可满足,从而验证规则库是否一致。为了进一步说明一致性检验的过程,我们可以将其分解为以下几个步骤:

\begin{itemize}
    \item \textbf{布尔变量的赋值:} 为规则库中的每个基本事件分配一个布尔变量,这些变量的真值表示事件是否发生。
    \item \textbf{布尔表达式的构建:} 利用布尔逻辑连接词(如与 $\land$、或 $\lor$、非 $\neg$)将基本事件的布尔变量组合成布尔表达式,以反映规则之间的逻辑关系。
    \item \textbf{合取范式的转换:} 将所有布尔表达式整合成一个合取范式(CNF),这是一个由多个子句构成的逻辑表达式,每个子句由一些变量的析取(或)组成。
    \item \textbf{可满足性求解:} 使用SAT求解器(如Z3)对CNF进行求解,以确定是否存在一组变量赋值,使得整个CNF为真。
    \item \textbf{结果分析:} 如果SAT求解器找到了一个解,则说明规则库是一致的;如果没有找到解,则表明规则库中存在冲突,需要进一步分析和调整。
\end{itemize}

在实际应用中,一致性检验不仅可以帮助我们确保规则库的逻辑一致性,还可以作为优化和改进规则库的依据。例如,如果发现某些规则之间存在冲突,我们可以通过修改或重新定义这些规则来解决冲突,从而提高规则库的整体质量和有效性。

总之,一致性检验是确保规则库可靠性和有效性的重要工具。通过将其转化为SAT问题并利用现代求解器的强大能力,我们可以有效地检测和解决规则库中的潜在冲突,为构建复杂系统和服务提供坚实的基础。

\cleardoublepage

\section{自动化}


















\par 我们可以用includegraphics来插入现有的jpg等格式的图片,如\autoref{fig:zju-logo}。

\begin{figure}[ht]
    \centering
    \includegraphics[width=.4\linewidth]{logo/zju}
    \caption{\label{fig:zju-logo}浙江大学LOGO}
\end{figure}

\par 如\autoref{tab:sample}所示,这是一张自动调节列宽的表格。

\begin{table}[ht]
    \caption{\label{tab:sample}自动调节列宽的表格}
    \begin{tabularx}{\linewidth}{|c|X<{\centering}|}
        \hline
        第一列 & 第二列 \\ \hline
        xxx & xxx \\ \hline
        xxx & xxx \\ \hline
        xxx & xxx \\ \hline
    \end{tabularx}
\end{table}

\par 如\autoref{equ:sample},这是一个公式

\begin{equation}
    \label{equ:sample}
    A=\overbrace{(a+b+c)+\underbrace{i(d+e+f)}_{\text{虚数}}}^{\text{复数}}
\end{equation}

\par 如\autoref{code:sample}所示,这是一段代码。
计算机学院的代码样式可能与其他专业不同,
如有需要,可以从计算机学院专业模板中复制相关的代码样式设定。

\begin{lstlisting}[%
    language={C},
    caption={simple.c},
    label={code:sample}
]
#include <stdio.h>

int main(int argc, char *argv[])
{
    printf("Hello, zjuthesis\n");
    return 0;
}
\end{lstlisting}

\subsection{关于字体}

英文字体通常提供了粗体和斜体的组合,中文字体通常没有粗体或斜体,本模板使用了 `AutoFakeBold' 来实现中文伪粗体,但不提供中文斜体,如\autoref{tab:font-examples}所示。

\begin{table}
    \centering
    \caption{一些字体示例}
    \label{tab:font-examples}
    \begin{tabular}{|c|c|c|c|c|}
        \hline
        字体            & 常规             & 粗体                       & 斜体                      & 粗斜体                                \\ \hline
        Times New Roman & Regular         & {\bfseries          Bold} & {\itshape         Italic} & {\bfseries \itshape      BoldItalic} \\ \hline
        仿宋            & {\fangsong 常规} & {\fangsong \bfseries 粗体} & {\fangsong \itshape 斜体} & {\fangsong \bfseries \itshape 粗斜体} \\ \hline
        宋体            & {\songti   常规} & {\songti   \bfseries 粗体} & {\songti   \itshape 斜体} & {\songti   \bfseries \itshape 粗斜体} \\ \hline
        黑体            & {\heiti    常规} & {\heiti    \bfseries 粗体} & {\heiti    \itshape 斜体} & {\heiti    \bfseries \itshape 粗斜体} \\ \hline
        楷体            & {\kaishu   常规} & {\kaishu   \bfseries 粗体} & {\kaishu   \itshape 斜体} & {\kaishu   \bfseries \itshape 粗斜体} \\ \hline
    \end{tabular}
\end{table}

\sectionnonum[none]{同一页上的章标题}
