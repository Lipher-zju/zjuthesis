\cleardoublepage{}
\begin{center}
    \bfseries \zihao{3} 摘~要
\end{center}

随着信息技术的飞速发展,我们进入了一个信息和服务爆炸式增长的“大模型”时代。这一时代的到来极大地丰富了应用和服务的多样性,但同时也带来了前所未有的监管挑战。在这样的背景下,本研究探索并设计一种统一、规范的服务监管语言,不仅可以为服务监管提供语言规范,而且规范化的语言更有利于识别与计算,能提高服务监管的效率和效果,同时实现对信息和服务复杂性与动态性的更好理解和控制。

在研究过程中,首先对现有的服务监管规则内容进行了深入的理论分析,提出了一套监管语言的语法设计,通过Antlr4和大模型实现了语法解析器。接着对监管语言进行了语义设计,将其构建为随机量化形式的SSAT问题,可用求解器进行可满足性计算。此外,该语言还使用SBERT进行相似性检验,并使用Z3-Solver进行一致性检验,以优化服务监管模型的检测和计算效率,确保规则库的质量。最后,本研究考虑了监管语言与大模型的适配性,通过一系列实验测试和验证了所设计的监管语言及其与大模型的契合效果。

研究结果表明,本研究所提出的服务监管语言能够有效地描述服务监管的各种规则,支持复杂的各领域规则,并且在实际应用中表现出了较高的准确性和效率。通过与大模型的适配设计,服务监管语言能够更好地服务于服务监管工作,为解决当前信息和服务爆炸式增长带来的挑战提供了有效的工具和策略。本研究的成果不仅为服务监管领域提供了新的理论和方法,也为相关领域的研究和实践提供了一定的参考。

\vspace{1\baselineskip}
\noindent \textbf{关键词:}服务监管、语言设计、大模型、智能识别

\cleardoublepage{}
\begin{center}
    \bfseries \zihao{3} Abstract
\end{center}

With the rapid development of information technology, we have entered an era of "large model" with explosive growth of information and services. The advent of this era has greatly enriched the diversity of applications and services, but it also brings unprecedented regulatory challenges. Against this backdrop, this study explores and designs a unified, standardized service regulation language, which not only provides language norms for service regulation, but also facilitates identification and computation with standardized language, enhancing the efficiency and effectiveness of service regulation, and achieving better understanding and control of the complexity and dynamics of information and services.

In this study, we first conducted a thorough theoretical analysis of the existing service regulation rule content, proposed a syntax design for the regulatory language, and implemented a syntax parser through Antlr4 and the large model. Then, we designed the semantics of the regulatory language, constructing it as a SSAT problem in the form of random quantification, which can be calculated for satisfiability by a solver. In addition, the language also uses SBERT for similarity testing, and Z3-Solver for consistency testing, to optimize the detection and calculation efficiency of the service regulation model and ensure the quality of the rule library. Finally, this study considered the compatibility of the regulatory language with the large model, and tested and verified the designed regulatory language and its fit with the large model through a series of experiments.

The research results show that the service regulation language proposed in this study can effectively describe various rules of service regulation, support complex rules in various fields, and demonstrate high accuracy and efficiency in practical applications. Through adaptive design with the large model, the service regulation language can better serve the work of service regulation, providing effective tools and strategies to address the challenges brought by the current explosive growth of information and services. The achievements of this study not only provide new theories and methods for the field of service regulation, but also provide a reference for research and practice in related fields.

\vspace{1\baselineskip}
\noindent \textbf{Keywords:} Service Regulation, Language Design, Large model, Intelligent Recognition