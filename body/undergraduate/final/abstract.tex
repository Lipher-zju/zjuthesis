\cleardoublepage{}
\begin{center}
    \bfseries \zihao{3} 摘~要
\end{center}

随着信息技术的飞速发展,我们进入了一个信息和服务爆炸式增长的“大模型”时代。这一时代的到来,虽然极大地丰富了应用和服务的多样性,但同时也带来了前所未有的监管挑战。在这样的背景下,本研究旨在探索并设计一种有效的服务监管语言,以提高服务监管的效率和效果,同时实现对信息和服务复杂性与动态性的更好理解和控制。本研究的意义在于,通过设计一种统一、规范的服务监管语言,不仅可以为服务监管提供语言规范,而且规制化的语言更有利于识别与计算。
在研究过程中,我们首先对现有的服务监管规则内容进行了深入的理论分析,提出了一套监管语言的语法设计,并通过Antlr4工具实现了语法解析器。接着,我们对监管语言进行了语义设计,将其构建为随机量化形式的SSAT问题,并采用SSAT求解器进行可满足性计算。此外,我们还设计了相似性检验和一致性检验机制,以优化服务监管模型的检测和计算效率,并确保规则库的质量。最后,我们考虑了监管语言与大模型的适配性,通过一系列实验测试和验证了所设计的监管语言及其与大模型的契合效果。
研究结果表明,所提出的服务监管语言能够有效地描述服务监管的各种规则,支持复杂的表达式,并且在实际应用中表现出了较高的准确性和效率。通过与大模型的适配设计,我们的监管语言能够更好地服务于服务监管工作,为解决当前信息和服务爆炸式增长带来的挑战提供了有效的工具和策略。本研究的成果不仅为服务监管领域提供了新的理论和方法,而且为相关领域的研究和实践提供了宝贵的参考。

\cleardoublepage{}
\begin{center}
    \bfseries \zihao{3} Abstract
\end{center}
