\cleardoublepage

\section{绪论}

\subsection{背景}

随着信息技术的迅猛发展,数字内容和服务的规模、复杂性和动态性都在快速增长,这不仅为社会带来了前所未有的便利,也带来了一系列监管上的挑战。

首先,服务监管\cite{服务监管}的复杂性日益增加。服务规则内容广泛,涉及经济、社会、文化等多个领域,不同领域的用语差异巨大,加之不同撰写者的习惯和多语种风格的差异,使得规则内容的统一和标准化变得十分困难。此外,服务的动态性也给监管带来了挑战。在数字化时代,服务的更新换代速度极快,新的服务模式和服务形态不断涌现,这对监管规则的及时更新和适应性提出了更高的要求。

其次,传统的服务监管方法已经难以适应当前的形势。随着服务规模的扩大和复杂性的增加,传统的监管方式在效率和效果上都显得力不从心。如何利用大模型在数据处理、智能识别以及动态调整等方面的超强能力,研制出基于大模型的服务监管方法和平台,实现高效监管,成为亟待解决的问题。

此外,监管语言的设计也面临着新的挑战。现有的监管语言设计相对粗浅,难以应对当前的问题。如何设计一种可泛化、易处理的监管语言,以支持各种复杂的表达式,方便用户进行规则的编写和修改,成为服务监管领域面临的一个重要课题。


\subsection{意义}

在信息爆炸和服务多样化的时代,设计一种专门用于服务监管的语言显得至关重要。随着数字化转型的步伐不断加快,人们面临着一系列前所未有的挑战:如何更加高效地处理和监管大量的信息服务、如何确保监管规则的一致性和可操作性、如何适应服务的快速迭代和更新。设计一种统一且规范的服务监管语言,不仅能提升服务监管效率和成果,而且能为监管者提供一种更精确和灵活的工具,应对日渐复杂的监管环境。

服务监管语言的设计是实现自动化和智能化监管的关键一步。通过将复杂的监管规则转化为一种可计算的语言,可以借助大模型等先进的数据处理技术,实现对服务的智能识别和动态调整,提升服务监管的响应速度和准确度。这种转变不仅能显著减轻监管者的工作负担,同时也有助于提高服务监管的质量和水平。

另外,设计服务监管语言的过程中,包含了多个领域的前沿技术,比如自然语言处理、逻辑编程以及概率模型等,这个过程本身就是一个推动相关学科发展和创新的过程。在实践中,它能提供一种全新的解决方案,帮助监管者更好地应对数字化时代的挑战,进一步保障信息服务的健康有序发展。

设计一种有效的服务监管语言,对于提高监管效率、保障信息安全、推动学科发展等方面都具有重要的意义。通过本研究,能为服务监管领域提供一种新的理论和方法,为解决当前面临的挑战做出贡献。

\subsection{全文架构}

本研究设计并实现了一种服务监管语言 \textbf{HORAE}\footnote{源自希腊神话中的秩序女神 \textit{Horeā}。} ,以支持服务监管系统。

全文架构如下:第1章绪论阐述了服务监管语言的研究背景与目标;第2章背景知识与技术,为研究提供了必要的技术支撑;第3章 HORAE 语法设计详细介绍了 HORAE 语法的设计原则、特性归纳与设计;第4章 HORAE 语义设计探讨了 HORAE 的语义建模以及相似性检验和一致性检验;第5章自动化语法分析描述了自动化语法分析方法 ParseGPT 的设计和实现;第6章实验与分析展示了数据集的构建和对 HORAE 语言进行的实验验证;第7章总结与展望总结了本文的研究成果,并对 HORAE 未来的发展进行了展望。

本文作者的主要工作在于:设计并实现了服务监管语言 HORAE,涵盖其语法和语义模型;设计了自动化语法分析方法ParseGPT,极大提升了自然语言向 HORAE 转换的效率;参与构建与标注 FRR-Eval\footnote{本项目的基准数据集,由作者与孙雨涛、赵康嘉、陈锦涛共同完成,在 6.1 FRR-Eval 数据集 中详细介绍。} 基准数据集,为模型的训练与评估提供了宝贵资源。参与设计并进行一系列实验\footnote{其中模型的微调训练由孙雨涛学长独立完成。},验证了 HORAE 的有效性和实用性。

值得注意的是,HORAE 现已投入 \textbf{服务智能监管共性理论与技术} \footnote{由中国国家重点研发计划资助,批准号为2022YFF0902600;由浙江省自然科学基金重点项目资助,批准号为LD24F020013;以及由浙江大学教育基金会的启真人才计划资助。通讯作者:尹建伟;陈明帅。}的使用,并额外产出了一篇论文 \textbf{ HORAE: A Domain-Agnostic Modeling Language
for Automating Multimodal Service Regulation}\footnote{即将投稿,本文作者也是其主要作者之一。}。