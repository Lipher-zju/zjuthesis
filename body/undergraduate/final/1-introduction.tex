\cleardoublepage

\section{绪论}

\subsection{背景}

随着信息技术的迅猛发展,数字内容和服务的规模、复杂性和动态性都在快速增长,这不仅为社会带来了前所未有的便利,也带来了一系列监管上的挑战。

首先,服务监管的复杂性日益增加。服务规则内容广泛,涉及经济、社会、文化等多个领域,不同领域的用语差异巨大,加之不同撰写者的习惯和多语种风格的差异,使得规则内容的统一和标准化变得十分困难。此外,服务的动态性也给监管带来了挑战。在数字化时代,服务的更新换代速度极快,新的服务模式和服务形态不断涌现,这对监管规则的及时更新和适应性提出了更高的要求。

其次,传统的服务监管方法已经难以适应当前的形势。随着服务规模的扩大和复杂性的增加,传统的监管方式在效率和效果上都显得力不从心。如何利用大模型在数据处理、智能识别以及动态调整等方面的超强能力,研制出基于大模型的服务监管方法和平台,实现高效监管,成为亟待解决的问题。

此外,监管语言的设计也面临着新的挑战。现有的监管语言设计相对粗浅,难以应对当前的问题。如何设计一种可泛化、易处理的监管语言,以支持各种复杂的表达式,方便用户进行规则的编写和修改,成为服务监管领域面临的一个重要课题。

综上所述,这是一个信息和服务爆炸式增长的新时代,服务监管面临着前所未有的挑战。如何设计一种有效的服务监管语言,利用大模型的超强能力,实现服务监管的自动化和智能化,是当前亟待解决的问题。这不仅关系到服务监管的效率和效果,也关系到信息时代社会治理的现代化进程。


\subsection{意义}

在信息爆炸和服务多样化的时代,设计一种专门用于服务监管的语言显得至关重要。随着数字化转型的步伐不断加快,人们面临着一系列前所未有的挑战:如何更加高效地处理和监管大量的信息服务、如何确保监管规则的一致性和可操作性、如何适应服务的快速迭代和更新。设计一种统一且规范的服务监管语言,不仅能提升服务监管效率和成果,而且能为监管者提供一种更精确和灵活的工具,应对日渐复杂的监管环境。

更进一步,服务监管语言的设计是实现自动化和智能化监管的关键一步。通过将复杂的监管规则转化为一种可计算的语言,可以借助大模型等先进的数据处理技术,实现对服务的智能识别和动态调整,提升服务监管的响应速度和准确度。这种转变不仅能显著减轻监管者的工作负担,同时也有助于提高服务监管的质量和水平。

另外,设计服务监管语言的过程中,包含了多个领域的前沿技术,比如自然语言处理、逻辑编程以及概率模型等,这个过程本身就是一个推动相关学科发展和创新的过程。在实践中,它能提供一种全新的解决方案,帮助监管者更好地应对数字化时代的挑战,进一步保障信息服务的健康有序发展。

设计一种有效的服务监管语言,对于提高监管效率、保障信息安全、推动学科发展等方面都具有重要的意义。通过本研究,能为服务监管领域提供一种新的理论和方法,为解决当前面临的挑战做出贡献。

\subsection{全文架构}

本文旨在设计并实现一种服务监管语言\textbf{HORAE}\footnote{源自希腊神话中的秩序女神 \textit{Horeā}} ,以支持高效、自动化的服务监管系统。在后续的研究中,统一使用 \textbf{HORAE } 来指代该服务监管语言。全文架构如下:

\begin{itemize}
    \item \textbf{第1章 绪论:} 介绍了服务监管语言的研究背景、意义以及本文的研究目标和工作。
    \item \textbf{第2章 背景知识与技术:} 概述了服务监管系统架构、Antlr语法分析器、大语言模型、随机布尔可满足性、Z3求解器和SBERT等关键技术,为后续研究打下基础。
    \item \textbf{第3章 HORAE 语法设计:} 详细介绍了HORAE语法的设计原则、特性归纳与设计,以及如何实现自动化语法分析。
    \item \textbf{第4章 HORAE 语义设计:} 探讨了HORAE的语义建模,包括相似性检验和一致性检验,为监管规则的自动化处理提供了理论支持。
    \item \textbf{第5章 自动化语法分析:} 描述了自动化语法分析方法ParseGPT的设计和实现,包括基本事件提取、逻辑关系提取和基本事件模式匹配。
    \item \textbf{第6章 实验与分析:} 展示了对HORAE语言进行的实验验证,包括数据集构建、模型性能评估和实验结果分析。
    \item \textbf{第7章 总结与展望:} 总结了本文的研究成果,并对HORAE未来的发展方向和改进空间进行了展望。
\end{itemize}

作者的主要工作包括:
\begin{itemize}
    \item 设计并实现了服务监管语言HORAE,包括其语法和语义模型。
    \item 开发了自动化语法分析方法ParseGPT,显著提高了从自然语言到形式化语言转换的效率。
    \item 参与构建与标注FRR-Eval基准数据集\footnote{本研究的基础数据集,将在第6章 实验与分析 中详细介绍},为模型训练和评估提供了丰富的资源。
    \item 进行了一系列的实验验证,证明了HORAE语言的有效性和实用性。
    \item 对HORAE的未来发展提出了具有前瞻性的建议和改进方向。
\end{itemize}

通过上述工作,本文不仅为服务监管领域提供了一种新的技术解决方案,也为相关研究领域的发展贡献了新的思路和方法。作者相信,HORAE语言的提出和实现,将对服务监管的自动化和智能化产生积极影响。